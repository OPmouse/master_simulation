% ----------------------------
% Abstract (japanese)
% ----------------------------
%\clearpage
\pagenumbering{roman}
%\setcounter{page}{0}
\begin{jabstract}
  コグニティブ無線を用いた周波数共用において,周波数の二次利用者(SU: Secondary User) は既存の周波数割り当てユーザ(PU: Primary User) への干渉を回避する必要がある.その中で自身の通信品質を確保するためには,正確な電波環境推定技術が重要である.筆者らは,これまで車載無線機やスマートフォンといった移動端末が観測した膨大な電波環境情報から構築される電波環境データベースを提案してきた.テレビ帯域を対象とした実証実験により,従来の距離減衰モデルに基づく手法と比較してPU の平均受信電力値の空間的な分布を精度良く推定できることを明らかにしている.しかし,これまではPU の通信状態のON/OFF 遷移を考慮せずに観測値を一意に平均化していた.そのため,無線LAN のように観測期間内に状態遷移する可能性のあるシステムについては,最終的な平均結果とON 状態の平均受信電力値に差が生じる恐れがあった.そこで本稿では,PU の通信状態の遷移を検出するセンシング手法を提案する.提案手法では,電波環境データベースに蓄積された統計情報と連携し,協調センシングによって電波環境を観測することで状態遷移を検出する.本手法により,通信を行なっている状態での受信電力値の取り出しが可能となり,結果としてPU が通信を行なっている状態での平均受信信号電力値を精度良く推定できる.本稿では特に,状態遷移時間の検出に焦点を当てたシミュレーション評価を行ない,その有効性を示す.
\end{jabstract}
%
%
%
% ----------------------------
% Abstract (english)
% ----------------------------
\begin{abstract}
Recently, with the fast development of wireless communication technology, cognitive radio (CR) has been recognized as a promising solution to address the problem of impending spectrum scarity for improving the utilization of spectrum for various wireless applications [1], [2]. In a CR system, it allows the Secondary Users (SUs) to opportunistically utilize the temporal and/or spatial unused spectrum holes without harmful interference to Primary Users (PUs). While SUs can occupy avaliable spectrum holes as long as the corresponding PU is in active, they must immediately evacuate the band as soon as the corresponding PU appears. One of the main chanllenges is to intelligently determine ongoing PU activity to avoid interferece toward PU. SUs can evacuate the band without affecting PU’s activity and opportunistically access the spectrum to maximize the spectrum usage if the information about PU can be obtained in advance. Hence, more information about PU leads to more effective spectrum usage for SUs, and an external device for provding information of PU is necessary. One of the main chanllenges is to intelligently determine ongoing PU activity to avoid interferece toward PU. SUs can evacuate the band without affecting PU’s activity and opportunistically access the spectrum to maximize the spectrum usage if the information about PU can be obtained in advance. Hence, more information about PU leads to more effective spectrum usage for SUs, and an external device for providing information of PU is necessary.
\end{abstract}
