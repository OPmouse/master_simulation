\thispagestyle{empty}
\begin{center}
  \vspace*{20mm}

  {\LARGE 平成27年度修士論文\\}

  \vspace{10mm}

  {\huge\bf Radio Environment Database considering Primary User Activity in Time Domain}

  \vspace{5mm}

  {\huge\bf プライマリユーザの時間的変化を考慮した\\電波環境データベース構築\\}

  \vspace{10mm}

  \begin{table}[h]
    \LARGE
    \begin{center}
      \begin{tabular}{ll}
        学生番号 & 1431019\\
        氏名 & 王 昊\\
        情報・通信工学専攻  & 情報通信システムコース\\
        主指導教員 & 藤井 威生 教授\\
        指導教員 & 山尾 泰 教授\\
        提出日 & 平成28年1月29日\\
      \end{tabular}
    \end{center}
  \end{table}

  \begin{table}[h]
    \Large
    \begin{center}
      \begin{tabular}{|c|c|c|}
        \hline
        \makebox[90pt][c]{専攻主任印} & \makebox[90pt][c]{主指導教員印}  & \makebox[90pt][c]{指導教員印} \\
        \hline
        &&\\
        &&\\
        &&\\
        &&\\
        \hline
      \end{tabular}
    \end{center}
  \end{table}
\end{center}

\clearpage
\thispagestyle{empty}
\begin{center}
  {\Large 修 士 論 文 の 和 文 要 旨}
  \renewcommand{\arraystretch}{1.5}
  \begin{table}[!h]

    \begin{tabular}{|c|c|c|c|}
      \hline
      研究科・専攻 & \multicolumn{3}{|c|}{大学院 情報理工学研究科 情報・通信工学専攻 博士前期課程}\\
      \hline
      \makebox[0.15\hsize][c]{氏    名} & \makebox[0.3\hsize][c]{王 昊} & \makebox[0.15\hsize][c]{学籍番号} & \makebox[0.3\hsize][c]{1431019} \\
      \hline
      論 文 題 目 \rule[-7mm]{0mm}{16mm} & \multicolumn{3}{|c|}{\parbox[c]{0.75\hsize}{Radio Environment Database considering Primary User Activity in Time Domain}} \\
      \hline
      \multicolumn{4}{|l|}{ 要  旨 } \\
      \multicolumn{4}{|l|}{\parbox[t]{\hsize}{
近年,周波数の枯渇問題を解消するための抜本的な対策として,周波数共用におけるコグニティブ無線技術が活発に研究を行われている.コグニティブ無線を用いた周波数共用において,周波数の二次利用者(SU: Secondary User) は既存の周波数割り当てユーザ(PU: Primary User)への干渉を回避する必要がある.その中で自身の通信品質を確保するためには,正確な電波環境推定技術が重要である.従来研究では,本研究室がこれまで車載無線機やスマートフォンといった移動端末が観測した膨大な電波環境情報から構築される実測値に基づく電波環境データベースを提案してきた.テレビ帯域を対象とした実証実験により,従来の距離減衰モデルに基づく手法と比較してPUの平均受信電力値の空間的な分布を精度良く推定できることを明らかにしている.しかし,これまではPUの通信状態のON/OFF遷移を考慮せずに観測値を一意に平均化していた.そのため,無線LANやセルラ通信のように観測期間内に状態遷移する可能性のあるシステムについては,最終的な平均結果とON状態の平均受信電力値に差が生じる恐れがあった.そこで本研究では,観測期間内にPUの通信状態が時間的に遷移する場合の電波環境データベースの構築における精度について検討を行う.1回の観測期間内での受信電力に関する分布変化を検出することにより,PU の通信状態の遷移点を検出するアルゴリズムを提案する.その分布の変化検出にCUSUMアルゴリズムとGLRアルゴリズムが用いられ,2つの分布の累積対数尤度比の変化傾向から遷移点の検出が可能となる.検出した遷移点を用いて,通信を行なっている状態のみの有効期間から受信電力値の取り出しが可能となり,PU が通信を行なっている状態での平均受信信号電力値を精度良く推定できる.本研究では特に,状態遷移の検出性能及び受信電力値の推定性能に焦点を当てたシミュレーション評価を行ない,その有効性を示す.シミュレーションにより,提案手法を用いることで遷移点が精度よく検出され,受信電力値の推定誤差が状態遷移が考慮されていない従来手法より抑えることが可能である.}} \\
      \multicolumn{4}{|l|}{}\\
      \multicolumn{4}{|l|}{}\\
      \multicolumn{4}{|l|}{}\\
      \multicolumn{4}{|l|}{}\\
      \hline
    \end{tabular}

  \end{table}
  \renewcommand{\arraystretch}{1.0}
\end{center}
