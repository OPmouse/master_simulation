\chapter[Introduction]{Introduction}
%\addcontentsline{toc}{chapter}{Chapter 1\\Introduction}
\label{chapter:introduction}
\pagenumbering{arabic}

In chapter \ref{chapter:introduction},  present spectrum scarcity problem as the background of this thesis and technology proposed for solution is described. Aslo, the overview of this thesis and purpose is described. 
\section{Background}
 Due to the rapid development of wireless communication systems, a demand on sprectrum resource for communication has increased explosively. Because the data rate and perfomance of the wireless communication system, such as mobile phone, are improved, it leads to a serious sprectrum scarcity problem.

 From Fig. \ref{fig:Cisco}, reference \cite{ref:Cisco} predicts that Global mobile data traffic will increase nearly tenfold between 2014 and 2019 and mobile data traffic will grow at a compound annual growth rate (CAGR) of 57 percent from 2014 to 2019, reaching 24.3 exabytes per month by 2019. 
\begin{figure}[!htp]
\includegraphics[width=150mm,clip]{traffic_trend.pdf}
\caption{Cisco Forecasts 24.3 Exabytes per Month of Mobile Data Traffic by 2019}
\label{fig:Cisco}
\end{figure}
In addition to the increasement of the data traffic, a fixed resource allocation method as the current spectrum allocation policy, which is utilized for avoiding harmful interference toward licensed systems with each other, is also considered as a major reason for the scarcity of the spectrum resource.
Since the finite spectrum resources are not able to fulfill the expoential growth of demand on traffic, it is necessary to review the present spectrum policy with fixed resource allocation for the next generation wireless communication sysytems and a effecient spectrum utilization turns to be a key problem. 
Therefore, the idea of Dynamic Spectrum Access(DSA) is attracted attendtion as effective soluation to the shortage of spectrum resouce. 





\section{Spectrum Sharing Trend and Problem}

\section{Purpose}


