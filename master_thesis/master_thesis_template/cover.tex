\documentclass[a4paper,11pt,openany]{book}
\setlength{\textheight}{\paperheight}   % 紙面縦幅を本文領域にする(BOTTOM=-TOP)
\setlength{\topmargin}{4.6truemm}       % 上の余白を30mm(=1inch+4.6mm)に
\addtolength{\topmargin}{-\headheight}  % 
\addtolength{\topmargin}{-\headsep}     % ヘッダの分だけ本文領域を移動させる
\addtolength{\textheight}{-60truemm}    % 下の余白も30mm(BOTTOM=-TOPだから+TOP+30mm)
\setlength{\textwidth}{\paperwidth}     % 紙面横幅を本文領域にする(RIGHT=-LEFT)
\setlength{\oddsidemargin}{0truemm}  % 左の余白を25mm(=1inch-0.4mm)に
\setlength{\evensidemargin}{0truemm} % 
\addtolength{\textwidth}{-50truemm}     % 右の余白も25mm(RIGHT=-LEFT)

\makeatletter
\pagestyle{empty}

\begin{document}

\begin{center}
  \vspace*{20mm}

  {\LARGE 平成27年度修士論文\\}

  \vspace{10mm}

  {\Huge\bf English Title\\}

  \vspace{5mm}

  {\huge\bf 日本語タイトル\\}

  \vspace{10mm}

  \begin{table}[h]
    \LARGE
    \begin{center}
      \begin{tabular}{ll}
        学生番号 & 1431xxx\\
        氏名 & \\
        情報・通信工学専攻  & 情報通信システムコース\\
        主指導教員 & 藤井 威生 教授\\
        指導教員 & 山尾 泰 教授\\
        提出日 & 平成28年1月29日\\
      \end{tabular}
    \end{center}
  \end{table}
  
\end{center}
\end{document}
